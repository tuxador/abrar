

\documentclass[headlines=6,headinclude=true,11pt]{scrartcl}
\usepackage{scrlayer-scrpage}
\usepackage[left=1.5cm,right=1.5cm,top=4cm,bottom=2cm,includefoot]{geometry}
\setlength{\footheight}{35pt}
\usepackage[usenames,dvipsnames,svgnames,table]{xcolor}
\usepackage{graphicx}
\usepackage[skins]{tcolorbox}
\usepackage{titlesec}
\usepackage{longtable}
\usepackage{booktabs}
\usepackage{fontawesome}
\usepackage{ifthen}
\usepackage{polyglossia}
\defaultfontfeatures{Mapping=tex-text}
\setdefaultlanguage{french}
\setmainfont[Mapping=tex-text]{Linux Libertine O}
\setsansfont{Roboto}
\setotherlanguage{arabic}
\newfontfamily\arabicfont[Script=Arabic]{Noto Kufi Arabic}
\definecolor{lightGris}{RGB}{185,185,185}
\definecolor{darkGris}{RGB}{115,115,115}
\definecolor{Gris}{RGB}{63,75,81}

\ihead{\begin{tabular}{c} \textcolor{Gris}{\textsf{\bfseries{Clinique médico chirurgicale EL ABRAR}}}\\
\textcolor{Gris}{\bfseries{\textarabic{العيادة الطبية الجراحية الأبرار}}}\\
         \textcolor{Gris}{\textsf{\bfseries{Cardiologie médicale et interventionnelle}}}\\\textcolor{Gris}{\bfseries{\textarabic{ مصلحة أمراض القلب}}}\\
   \end{tabular}
} 
%%%% eviter l'italic dans le header
%\renewcommand{\headfont}{\sffamily}
\chead{\includegraphics[height=5\baselineskip]{media/logo.png}}

\ohead{\begin{tabular}{r}\textcolor{Gris}{\textsf{\bfseries{Dr Yahyaoui Mohamed Kaddour}}}\\
         \textcolor{Gris}{\textsf{\bfseries{Cardiologue}}}\\ \textcolor{Gris}{\textsf{\bfseries{kaddourkardio{\faAt}gmail.com}}}\end{tabular}}



\cfoot{\textcolor{Gris}{\textit{\textbf{``Votre coeur mérite ce qu'il y a de
        mieux''} } }\\
\vspace{-3mm}
\hrulefill\\
\begin{minipage}{0.32\textwidth}
\textcolor{Gris}{\sffamily{ {\faHospitalO} 15,rue Tolozane} }\\
\textcolor{Gris}{\sffamily{Les palmiers, Oran} }
\end{minipage}
\hfill
\begin{minipage}{0.25\textwidth}
 \textcolor{Gris}{\sffamily{\sffamily{ {\faPhone} (+213) 041-496-066} } } \\
 \textcolor{Gris}{\sffamily{ {\faFax} (+213) 041-233-491} }
\end{minipage}
\hfill
\begin{minipage}{0.27\textwidth}
\textcolor{Gris}{\sffamily{\sffamily{ {\faMobile} (+213) 0561 882-066}} } \\
\textcolor{Gris}{\sffamily{\small{\faEnvelopeO} {clinique-elabrar{\faAt}laposte.net} }}
\end{minipage} }


\begin{document}

\begin{center}
  \textcolor{Gris}{\bfseries{\huge{\textsf{RAPPORT MÉDICAL}}} }
\end{center}

  \section*{Patient}
  \begin{description}
	\item[Nom prénoms:] {{consultation.patient}}
	\item[Adresse:] {{patient.adresse}}
	\item[Date de naissance:] {{patient.birth}}
	\item[Motifs de consultation:] Consultation dans le cadre de l'urgence pour Consultation ce jour le {{ consultation.consultation_date }} pour {{ motif }} . 

    \end{description}
    \subsection*{Facteurs de risques - antécédents }
      \begin{description}
        
      \item[FDR:]
        
        \begin{itemize}
           \item Hypertension artérielle.
             \item Diabète. 
               \item Dyslipidémie. 
                 \item Tabagique. 
                   \item Hérédité coronarienne. 
                     \item Obésité.
                      
                    \end{itemize}

                  \item[Antécédents:]
                    {{ patient.history }}
                  \item[Allergies connues:] {{ patient.allergy }}
                  \end{description}
                  \subsection*{Histoire de la maladie}
                  {{ consultation.histoire }}
                  
                  % \begin{center}
                  %   \rule{8cm}{0.4pt}
                  % \end{center}


                  \subsection*{Examen clinique}
                  \subsubsection*{Signes fonctionnels}
                  \begin{itemize}
                    \item fièvre.
                      \item Eupnéique au repos\item Dyspnée au stade {{ consultation.dyspnea_nyha }} de la \textit{NYHA}.
                        \item Absence d'angor.\item Angor au stade {{ consultation.angina_scc }} de la \textit{SCC}.
      \item Syncope.

\end{itemize}
\subsubsection*{Signes physiques}
 L'examen clinique retrouve à l'auscultation: {{ consultation.auscultation }} avec
 une fréquence cardiaque à = {{ consultation.heart_rate}} Bpm.\\

 En périphérie les pouls sont {{ consultation.pulse }} avec une PA = {{ consultation.systolic_bp }}/{{ consultation.diastolic_bp }}mmHg.\\

\section*{Bilan para clinique}
\subsection*{ECG}
Le rythme est en sinusalfibrillation atrialeflutter auriculairetacchycardie ventriculaireélectro stimulé
          à fréquence cardiaque de {{ consultation.freq }} puls/min.\\

  Il existe un bloc de branche gauche.\\Il existe un bloc de branche droit.\\
    On note aussi une hypertrophie ventriculaire gauche électrique.\\
      Sur le plan ischémique:  une lésion en septal une lésion en antérieur une lésion en latéral une lésion en postérieurabsence de lésions électriques,
               une ischémie en septal une ischémie en antérieur une ischémie en latéral une ischémie postérieurabsence d'ischémie électrique,
                       onde T inversée en septal onde T inversée en antérieur onde T inversée en en latéral onde T inversée en postérieur. \\
            Le QT corrigé est estimé à {{ consultation.corrected_qt }} msec. 
            

 \subsection*{Téléthorax}
Le téléthorax retrouve: {{ consultation.telethorax }} \\

             
\subsection*{Échocardiographie du {{ consultation.date_echo }} }
                
À l'écho cardiographie on a : {{consultation.echocoeur }}.\\
               
 \textbf{l'ETO} retrouve: {{consultation.eto }}

 \subsection*{Biologie}

voir bilans. \\
                  
\section*{Én résumé}
                  
{{ consultation.resume }}
                  
\subsection*{Traitement}
                  
{{consultation.ordonnance}}
                  
\subsection*{Dispositions complémentaires}
{{ consultation.dispositions }}
                  \\
                  
 \begin{center}
  \rule{8cm}{0.4pt}
\end{center}
\vspace{2cm}
\flushbottom\raggedleft Oran le \today\\
\raggedleft Dr \textsc{Yahyaoui M K}\\
\end{document}
                
                
                
                  
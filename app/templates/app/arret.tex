

  \documentclass[12pt,a4paper]{article}
  \usepackage[usenames,dvipsnames,svgnames,table]{xcolor}
  \usepackage{graphicx}
  \usepackage[skins]{tcolorbox}
  \usepackage{titlesec}
  \usepackage{longtable}
  \usepackage{booktabs}
  \usepackage{xltxtra}
  \usepackage{xunicode}
  \usepackage{fontspec}
  \defaultfontfeatures{Mapping=tex-text}
  \usepackage[a4paper,left=2cm,right=2cm,top=1.5cm,bottom=2cm]{geometry}
  \usepackage{fancyhdr}
  \usepackage{polyglossia}
  \setdefaultlanguage{french}
  \setmainfont[Mapping=tex-text]{Linux Biolinum O}
  \setsansfont{Linux Biolinum}
  \setotherlanguage{arabic}
  \newfontfamily\arabicfont[Script=Arabic]{Mothanna}

  % définir \tightlist pour la compatibilité avec pandoc 1.14
  \providecommand{\tightlist}{%
	 	\setlength{\itemsep}{0pt}\setlength{\parskip}{0pt}}
  % \usepackage{amsmath}
  % \usepackage{amsfonts}
  % \usepackage{amssymb}

  % \colorlet{sectioncolor}{orange}
  % \makeatletter
  % \renewcommand\sectionlinesformat[4]{%
  % \colorbox{sectioncolor}{%
  % \parbox[t]{\dimexpr\textwidth-2\fboxsep\relax}{%
  % \raggedsection\color{white}\@hangfrom{#3}{#4}%
  % }}}
  %  \makeatother


  \newtcolorbox{imagetextbox}[3][]{%
    before=\par\bigskip\noindent,after=\par\medskip,
    blank,sidebyside,center lower,
    fontlower=\fontsize{25pt}{28pt}\selectfont\bfseries,
    width=\textwidth-#2-#3,
    lefthand width=#2,
    sidebyside gap=#3,#1}

  \newcommand{\imagetext}[3][0.15]{%
    \begin{imagetextbox}[]{#1\textwidth}{3mm}%
      \includegraphics[width=\linewidth]{#2}%
      \tcblower%
	    #3%
    \end{imagetextbox}}

  \usepackage{tikz}\usetikzlibrary{shapes.misc}
  \newcommand\titlebar{%
    \tikz[baseline,trim left=3.1cm,trim right=3cm] {
      \fill [cyan!25] (2.5cm,-1ex) rectangle (\textwidth+3.1cm,2.5ex);
      \node [
      fill=cyan!60!white,
      anchor= base east,
      rounded rectangle,
      minimum height=3.5ex] at (3cm,0) {
        \textbf{\thesection.}
      };
    }%
  }
  \titleformat{\section}{\large}{\titlebar}{0.1cm}{}
  \renewcommand*{\thesection}{\arabic{section}}
  \pagestyle{fancy}
  \usepackage{bidi}
  \author{Dr Yahyaoui mohamed Kaddour}
  \title{Rapport médical }
\setlength\parindent{0pt}
  \begin{document}
  \lhead{Clinique \textsc{El Abrar}}
  \rhead{Rapport médical}

  \imagetext{media/logo.png}{\textcolor{Gray}{\large{\textarabic{العيادة الطبية الجراحية الأبرار}}\\
      Clinique Médico-Chirurgicale \textsc{El Abrar}\\
      \textarabic{ مصلحة أمراض القلب}\\
      \small{Service de cardiologie médicale et interventionnelle}
    }}
  \begin{center}
    \textcolor{gray}{\textbf{ Dr \textsc{Yahyaoui} Mohamed Kaddour\\
        Spécialiste en cardiologie}\\
    }
  \end{center}

  \begin{center} \huge{Rapport médical}

  \end{center}


  \section{Patient}
  \begin{description}
	\item[Nom prénoms:] {{ consultation.patient }}
	\item[Adresse:] {{ patient.ardress }}
	\item[Date de naissance:] {{ patient.birth }}
	\item[Motifs de consultation:] Consultation dans le cadre de l'urgence pour Consultation ce jour le {{ consultation.consultation_date }} pour {{ motif }} . 

    \end{description}
         \section{Facteurs de risques - antécédents }
      \begin{description}
        
      \item[FDR:]
        
        \begin{itemize}
           \item Hypertension artérielle.
             \item Diabète. 
               \item Dyslipidémie. 
                 \item Tabagique. 
                   \item Hérédité coronarienne. 
                     \item Obésité.
                      
                    \end{itemize}

                  \item[Antécédents:]
                    {{ patient.history }}
                  \item[Allergies connues:] {{ patient.allergy }}
                  \end{description}
                  \section{Histoire de la maladie}
                  {{ consultation.histoire }}
                  
                  % \begin{center}
                  %   \rule{8cm}{0.4pt}
                  % \end{center}


                  \section{Examen clinique}
                  \subsection{Signes fonctionnels}
                  \begin{itemize}
                    \item fièvre.
                      \item Eupnéique au repos\item Dyspnée au stade {{ consultation.dyspnea_nyha }} de la \textit{NYHA}.
                        \item Absence d'angor.\item Angor au stade {{ consultation.angina_scc }} de la \textit{SCC}.
      \item Syncope.

\end{itemize}
\subsection{Signes physiques}
 L'examen clinique retrouve à l'auscultation: {{ consultation.auscultation }} avec
 une fréquence cardiaque à = {{ consultation.heart_rate}} Bpm.\\

 En périphérie les pouls sont {{ consultation.pulse }} avec une PA = {{ consultation.systolic_bp }}/{{ consultation.diastolic_bp }}mmHg.\\

\section{Bilan para clinique}
\subsection{ECG}
Le rythme est en sinusalfibrillation atrialeflutter auriculairetacchycardie ventriculaireélectro stimulé
          à fréquence cardiaque de {{ consultation.freq }} puls/min.\\

  Il existe un bloc de branche gauche.\\Il existe un bloc de branche droit.\\
    On note aussi une hypertrophie ventriculaire gauche électrique.\\
      Sur le plan ischémique:  une lésion en septal une lésion en antérieur une lésion en latéral une lésion en postérieurabsence de lésions électriques,
               une ischémie en septal une ischémie en antérieur une ischémie en latéral une ischémie postérieurabsence d'ischémie électrique,
                       onde T inversée en septal onde T inversée en antérieur onde T inversée en en latéral onde T inversée en postérieur. \\
            Le QT corrigé est estimé à {{ consultation.corrected_qt }} msec. \\
            
            
              \subsection{Téléthorax}
              Le téléthorax retrouve: {{ consultation.telethorax }} \\
              
              
                
                \subsection{Échocardiographie du {{ consultation.date_echo }} }
                
                À l'écho cardiographie on a : {{consultation.echocoeur }}.\\
                
                 \textbf{l'ETO} retrouve: {{consultation.eto }}
                  
                  \subsection{Biologie}

                  % \begin{description}aspect de cardiopathie ischémique à fonction VG préservée \textbf{FEVG à 45\% Au Simpson biplan }.\\
                  %   Hypokinésie septale et akinésie inférieure.\\
                  %   Pas de vice valvulaire.\\
                  %   PAPS à 25mmHg.
                  % \item[Troponines] négatives.
                  % \item[FNS] HB = 11g/dl. plt = 178.000/mm3 GB = 6400/mm3
                  % \item[Sérologies] négatives.
                  % \item[Biochimie] glycémie = 0.98 g/l urée = 0.25g/l Créat = 8 mg/l.
                  % \end{description}
                  voir bilans. \\
                  
                  \section{Én résumé}
                  
                  {{ consultation.resume }}
                  
                  \subsection{Traitement}
                  
                  {{consultation.ordonnance}}
                  
                  \subsection{Dispositions complémentaires}
                  {{ consultation.dispositions }}
                  \\
                  
                  \begin{center}
                    \rule{8cm}{0.4pt}
                  \end{center}
                  \vspace{2cm}
                  \flushbottom\raggedleft Oran le \today\\
                  \raggedleft Dr \textsc{Yahyaoui M K}\\
                  \fancyfoot[L]{Dr \textsc{Yahyaoui M K}}
                \end{document}
                
                
                
                  